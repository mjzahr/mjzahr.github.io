\documentclass[10pt]{article}

% Packages
\usepackage{amssymb,amsmath,amsfonts,amsthm,mathtools}
\usepackage{paralist}
\usepackage[margin=1in]{geometry}
\usepackage{fancyhdr} % Required for custom headers
\usepackage{lastpage} % Required to determine the last page for the footer
\usepackage{extramarks} % Required for headers and footers
\usepackage[explicit]{titlesec}

% Set up the header and footer
\pagestyle{fancy}
\chead{\large\textbf{Advisor-Advisee Expectations}}
\lfoot{\lastxmark} % Bottom left footer
\cfoot{}
\rfoot{Page\ \thepage\ of\ \protect\pageref{LastPage}} % Bottom right footer
\renewcommand\headrulewidth{0.4pt} % Size of the header rule
\renewcommand\footrulewidth{0.4pt} % Size of the footer rule

% Modify section, subsection styles
\titleformat{\section}{\normalfont\normalsize\bfseries}{\thesection}{}{\underline{#1}}
\titlespacing*{\section}{0pt}{8pt}{2pt}
\titleformat{\subsection}{\normalfont\normalsize\sc}{\thesubsection}{}{#1}
\titlespacing*{\subsection}{0pt}{8pt}{2pt}
\setlength\parindent{0pt}

\begin{document}

Read each statement describing an aspect of the advisor-advisee relationship
and determine whether you believe this to be the responsibility of the
advisor, the advisee, or somewhere in between. Indicate your position on
a scale of 1 to 5, where 1 indicates this is solely the reponsibility of
the advisor, 5 indicates this is solely the responsibility of the advisee,
and 3 indicates the advisor and advisee equally share this responsibility.
Include comments to provide additional detail on your position. Some aspects
to do not fit cleanly into the described format so both ends of the continuum
are included the statement, along with their position on the scale.%
\footnote{
The content of this document is largely taken from a document received
from John Boothroyd. Original from Ingrid Moses, 1985, Higher Education
Research and Development Society of Australasia. Adapted by Margaret Kiley
and Kate Cadman, 1997, Centre for Learning \& Teaching, Univ. of Technology,
Sydney. Further adapted by Chris M. Golde, 2010, Stanford University.
}

\section*{Course of Study and Dissertation Planning}

\subsection*{Selection of courses the student takes}
\textit{Scale}: % TODO: Additional comments here

\textit{Comments}: % TODO: Additional comments here

\subsection*{Selection of the student's dissertation topic}
\textit{Scale}: % TODO: Additional comments here

\textit{Comments}: % TODO: Additional comments here

\subsection*{Selection of the members of the student's
             dissertation reading committee}
\textit{Scale}: % TODO: Additional comments here

\textit{Comments}: % TODO: Additional comments here


\section*{Contact and Involvement}
\subsection*{Determine when the advisor and student should meet}
\textit{Scale}: % TODO: Additional comments here

\textit{Comments}: % TODO: Additional comments here

\subsection*{The advisor should check regularly that the student is making
             progress (1) vs. the student should work independently without
             having to account for how they spend their time (5)}
\textit{Scale}: % TODO: Additional comments here

\textit{Comments}: % TODO: Additional comments here

\subsection*{The advisor should be the first place to turn when the student
             has problems with the research project (1) vs. students should
             try to resolve problems on their own, including seeking input
             from others, before bringing a research problem to the advisor
             (5)}
\textit{Scale}: % TODO: Additional comments here

\textit{Comments}: % TODO: Additional comments here

\subsection*{Emotional support and encouragement to the student}
\textit{Scale}: % TODO: Additional comments here

\textit{Comments}: % TODO: Additional comments here


\section*{Dissertation}
\subsection*{The advisor should insist on seeing all drafts to ensure the
             student is on the right track (1) vs. students should submit
             drafts of work only when they want input and feedback from the
             advisor (5)}
\textit{Scale}: % TODO: Additional comments here

\textit{Comments}: % TODO: Additional comments here

\subsection*{Writing of the dissertation}
\textit{Scale}: % TODO: Additional comments here

\textit{Comments}: % TODO: Additional comments here

\subsection*{Determine when and where to present or publish the research}
\textit{Scale}: % TODO: Additional comments here

\textit{Comments}: % TODO: Additional comments here

\subsection*{Decide when the dissertation is ready to be defended and
             submitted}
\textit{Scale}: % TODO: Additional comments here

\textit{Comments}: % TODO: Additional comments here

\subsection*{Responsibility of the quality of the dissertation}
\textit{Scale}: % TODO: Additional comments here

\textit{Comments}: % TODO: Additional comments here

\section*{Support}
\subsection*{Finding funding for the student}
\textit{Scale}: % TODO: Additional comments here

\textit{Comments}: % TODO: Additional comments here

\subsection*{Building the student's network}
\textit{Scale}: % TODO: Additional comments here

\textit{Comments}: % TODO: Additional comments here

\subsection*{Career advice and preparation}
\textit{Scale}: % TODO: Additional comments here

\textit{Comments}: % TODO: Additional comments here

\section*{Advisor-advisee expectations agreement}
By signing this document, you agree you have completed this document
truthfully and to the best of your ability.

% TODO: Add name and sign (either electronically or print/sign/scan)
\begin{itemize}[]
 \setlength\itemsep{3mm}
 \item \today, ~~ Printed Name
\end{itemize}

\end{document}
